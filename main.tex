\documentclass[12pt,a4paper]{report}

\usepackage{graphicx} % Required for inserting images
\usepackage{amsmath,amsfonts,amssymb} % must required for your math type
\usepackage{inputenc}
\usepackage{wallpaper}
\usepackage{tikz} % plot edit
\usepackage{xcolor}
\usepackage{geometry}% typographical
\usepackage{color}
\usepackage{units}
\usepackage{fontspec} % 字体包
\usepackage{unicode-math} % 数学字体包
\usepackage[most]{tcolorbox}
\usepackage{fvextra}
\usepackage{fancyhdr} % fncychap(various of styles of chapters)
\usepackage{tcolorbox}
\tcbuselibrary{skins, breakable, theorems}
\usepackage{bm}


% 定义一个新的计数器,并让它依赖于章节计数器
\newcounter{definitioncounter}[section]
% 定义编号格式,例如:1.1, 1.2, 2.1, 等等
\renewcommand{\thedefinitioncounter}{\thesection.\arabic{definitioncounter}}
\newcommand{\definition}[1]{%
  \refstepcounter{definitioncounter}% 增加计数器,并使其值可引用
  \textcolor{deepblue}{\sffamily\large\textbf{Definition \thedefinitioncounter: #1}}% 显示带编号的标题
}

\newtheorem{lemma}{\textit{lemma}}[section]
\newtheorem{inference}{Inference}[section]
\newtheorem{axiom}{Axiom}[section]


\newcounter{theoremcounter}[section]
\renewcommand{\thetheoremcounter}{\thesection.\arabic{theoremcounter}}
\newcommand{\theorem}[1]{%
  \refstepcounter{theoremcounter}
  \textcolor{theoremcolor!50!black}{\sffamily\large\textbf{Theorem \thetheoremcounter: #1}}
}

\newcounter{examplecounter}[chapter]
\renewcommand{\theexamplecounter}{\thechapter.\arabic{examplecounter}}
\newcommand{\example}{
  \refstepcounter{examplecounter}
  \textcolor{red!50!black}{\sffamily\large\textbf{Example \theexamplecounter}}
}



\cfoot{\thepage} % controled by package fancyhdr



\setmainfont{Times New Roman} %text font
\setmathfont{XITS Math} %math font



%where I put all my predefined stuff
\definecolor{definecolororeage}{RGB}{255, 126, 33}
\definecolor{yellow}{RGB}{255, 241, 181}
\definecolor{deepblue}{RGB}{10, 120, 255}
\definecolor{theoremcolor}{RGB}{23, 135, 163}
\def\eq1{e^{i\theta}=\cos(\theta)+i\sin(\theta)}

% 这边定义了一个文本分割线
\newcommand{\customline}[4][center]{
    \noindent % 段落不缩进
    \ifx#1left
        \makebox[0pt][l]{\textcolor{#4}{\rule{#2}{#3}}} % 左对齐
    \else\ifx#1right
        \makebox[\textwidth][r]{\textcolor{#4}{\rule{#2}{#3}}} % 右对齐
    \else
        \makebox[\textwidth]{\textcolor{#4}{\rule{#2}{#3}}} % 置中
    \fi\fi
}


%from here all above is preamble 

%where I do all the type begin
\begin{document}


%here is your title page

\newgeometry{top=0cm,left=0cm,right=0cm}
\begin{titlepage}
  %you can use \maketitle make title author date displace directly
  \centering
  % Insert the image
  \centerline{\includegraphics[width=\textwidth]{material/gochiusa_april2024.jpg}}%put any picture you prefer
  \customline[center]{\textwidth}{25pt}{definecolororeage} % argument: align style (left/center/right), length, width, color
  \vspace*{1cm} % Adjust space between the image and the title
   \begin{flushleft}
    \hspace*{2cm} 
    \begin{minipage}{\textwidth} 
        \Huge\textbf{\textrm{Here is your title}} 
        \vspace{1cm} 
        \par 
        \huge{here is your university} 
        \par 
        \huge{here is your department} 
    \end{minipage}
  \end{flushleft}
  \vfill
  \begin{flushleft}
    \hspace{2cm}
    \large{author:here is your name}
  \end{flushleft}
\end{titlepage}
\restoregeometry


\newgeometry{top=4cm}

%main body start here




\chapter*{Abstract}
here is abstract here is abstract here is abstract

\chapter*{Some useful formula}
test line $ \oint_{l} f(x,y) \textrm{d} \vec{l} $
\tableofcontents



%next generate new chapter files and use \include{chapter file}



\end{document} %where I do all the type end 


